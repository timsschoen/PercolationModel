\documentclass[a4paper,fleqn]{article}

%\usepackage{times}

\usepackage[margin = 2.5 cm]{geometry}
\usepackage[utf8]{inputenc}
\usepackage[T1]{fontenc}
\usepackage{tabularx}
\usepackage{graphicx}
\usepackage{subfigure}
\usepackage{tikz}
\usetikzlibrary{trees}
\usepackage{hyperref}
\usepackage{amsmath}
\usepackage{xcolor}
\usepackage{gensymb}
\usepackage{amsfonts}
\usepackage{natbib}
\usepackage{float}

\bibliographystyle{natdin}
\usetikzlibrary{positioning,shadows}

\begin{document}

\title{Percolation model\\ \Large{Report}}
\author{Tim Schön}
\date{}
\maketitle
\ \\

\section*{Percolation model}
For this model, only \emph{bond percolation}, as defined in \cite{Percolation}, was implemented and evaluated. The model is given as follows:
\section*{Grid types}
\subsection*{2D-grids}
\subsection*{3D-grids}
\section*{Implementation using \emph{Python} with \emph{PyQt}}
\bibliography{literature}

\end{document}