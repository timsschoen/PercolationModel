\documentclass[a4paper,fleqn]{article}

%\usepackage{times}

\usepackage[margin = 2.5 cm]{geometry}
\usepackage[utf8]{inputenc}
\usepackage[T1]{fontenc}
\usepackage{tabularx}
\usepackage{graphicx}
\usepackage{subfigure}
\usepackage{tikz}
\usetikzlibrary{trees}
\usepackage{hyperref}
\usepackage{amsmath}
\usepackage{xcolor}
\usepackage{gensymb}
\usepackage{amsfonts}
\usepackage{natbib}
\usepackage{float} 

\bibliographystyle{abbrv}
\usetikzlibrary{positioning,shadows}

\begin{document}

\title{Percolation model\\ \Large{Report}}
\author{Tim Schön}
\date{}
\maketitle
\ \\

\section*{Percolation model}
For this model, only \emph{bond percolation}, as defined in \cite{Percolation}, was implemented and evaluated. The model is given as follows:\\
Given a graph $G = (V,E)$ and a probability $p$, only consider an edge $e \in E$ \emph{active} with probability $p$ and inactive otherwise. 
What is the size of the largest connected component of the graph $S(p)$ only using active edges? 
How is this size dependent on the propability $p$ and the type of graph $G$?

\section*{Implementation using \emph{Python} with \emph{PyQt}}
To test and visualize this problem a small demo application was built. Different grids and gridsize can be chosen with a graphical 
user interface. The graph will then be visualized, after drawing active edges using the value $p$, which can be varied by the user.\\
Additionally, the application will find the connected components of the graph and calculate the fraction $\frac{S(p)}{|V|}$ for the largest connected component found. Breadth-first-search (BFS) is used to find the connected components for the subgraph of active edges.
To further visualize the percolation effect, a shortest path from the top to the bottom of the graph is found and visualized, if it exists. For this, the A* algorithm is used.
The value of $p$ and the corresponding $S(p)$ found by this analysis is added to a scatterplot of observations, which will be grown slowly as the user changes values of the sliders. 
The scatterplot can be cleared by the user.\\
\subsection*{Grid types}
The different grid types were programmed as different classes, each inheriting a common class (\verb|Graph|), which supports different essential algorithms for finding the connected 
components and shortest paths. Initially, the graphs are generated as a list of nodes as an numpy array with spatial coordinates and a list of edges, each storing the indices of the connected nodes.
To execute the A* and breadth-first-search algorithms the graph is transformed in adjacency-list-format.
\subsubsection*{2D-grids}
\subsubsection*{3D-grids}
\bibliography{literature}

\end{document}